
\section{Conclusions}

In this work, we presented a new approach to supervised dimensionality
reduction\textemdash one that attempts to learn orthogonal category
axes during training. The motivation for this work stems from the
observation that the semantics of the multi-class Fisher linear discriminant
are unclear especially w.r.t. defining a space for the categories
(classes). Beginning with this observation, we designed an objective
function comprising sums of quadratic and absolute value functions
(aimed at maximizing the inner product between each training set pattern
and its class axes) with Stiefel manifold constraints (since the category
axes are orthonormal). It turns out that recent work has characterized
such problems and provided sufficient conditions for the detection
of global minima (despite the presence of non-convex constraints).
The availability of a straightforward Stiefel manifold optimization
algorithm tailored to this problem (which has no step size parameters
to estimate) is an attractive by-product of this formulation. The
extension to the kernel setting is entirely straightforward. Since
the kernel dimensionality reduction approach warps the patterns toward
orthogonal category axes, this raises the possibility of using the
angle between each pattern and the category axes as a classification
measure. We conducted experiments in the kernel setting and demonstrated
reasonable performance for the angle-based classifier suggesting a
new avenue for future research. Finally, visualization of dimensionality
reduction for three classes showcases the category space geometry
with clear semantic advantages over principal components and multi-class
Fisher. 

Several opportunities exist for future research. We notice clustering
of patterns near the origin of the category space, clearly calling
for an origin margin (as in SVM's). At the same time, we can also
remove the orthogonality assumption (in the linear case) while continuing
to pursue multi-class discrimination. Finally, extensions to the multi-label
case \citep{sun2013multi} are warranted and suggest interesting opportunities
for future work.


